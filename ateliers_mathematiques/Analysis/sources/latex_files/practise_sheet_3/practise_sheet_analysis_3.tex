\documentclass[11pt, french]{article}
\usepackage{calc}
\usepackage{eso-pic}

\newlength{\PageFrameTopMargin}
\newlength{\PageFrameBottomMargin}
\newlength{\PageFrameLeftMargin}
\newlength{\PageFrameRightMargin}

\setlength{\PageFrameTopMargin}{1.5cm}
\setlength{\PageFrameBottomMargin}{1cm}
\setlength{\PageFrameLeftMargin}{1cm}
\setlength{\PageFrameRightMargin}{1cm}

\makeatletter

\newlength{\Page@FrameHeight}
\newlength{\Page@FrameWidth}

\AddToShipoutPicture{
  \thinlines
  \setlength{\Page@FrameHeight}{\paperheight-\PageFrameTopMargin-\PageFrameBottomMargin}
  \setlength{\Page@FrameWidth}{\paperwidth-\PageFrameLeftMargin-\PageFrameRightMargin}
  \put(\strip@pt\PageFrameLeftMargin,\strip@pt\PageFrameTopMargin){
    \framebox(\strip@pt\Page@FrameWidth, \strip@pt\Page@FrameHeight){}}}

\makeatother
%%%%%%%%%%%%%%%%%%%%%%%%%%%%%%%%%%%%%%%%%%%%%%%%%%%%%
% Importation des paquets
%%%%%%%%%%%%%%%%%%%%%%%%%%%%%%%%%%%%%%%%%%%%%%%%%%%%%
\usepackage[utf8]{inputenc}
\usepackage{helvet}
\usepackage{natbib}
\usepackage{graphicx}
\usepackage{libertine}
\usepackage{eso-pic}
\usepackage{hyperref} % allow to write hyperlink (open the url)
\usepackage{titlesec} % allow to change fontsize of the different section
\usepackage{tabularx} % allow to create tables with fixed length
\usepackage{listings} % code highlighting
\usepackage{xcolor}
\usepackage{amsmath, nccmath}
\usepackage{amssymb}

\usepackage[a4paper,margin=1in]{geometry}
\definecolor{LightGray}{gray}{0.9}


%%%%%%%%%%%%%%%%%%%%%%%%%%%%%%%%%%%%%%%%%%%%%%%%%%%%%
% Definition des commandes
%%%%%%%%%%%%%%%%%%%%%%%%%%%%%%%%%%%%%%%%%%%%%%%%%%%%%
% Permet de définir une image en arrière plan
\newcommand\BackgroundPic{%
\put(0,0){%
\parbox[b][\paperheight]{\paperwidth}{%
\vfill
\centering
\includegraphics[width=\paperwidth,height=\paperheight]{background-42-ai.png}%
\vfill
}}}

% Définition de la commande pour faire un snippet de code inline
\newcommand{\inlsnippet}[1]{\colorbox{gray!10}{\mbox{\textcolor{pink}{#1}}}}

% Modification of the style of hyperlink, to be visible
\hypersetup{
    colorlinks=true,
    linkcolor=blue,
    filecolor=magenta,      
    urlcolor=cyan,
}

%%%%%%%%%%%%%%%%%%%%%%%%%%%%%%%%%%%%%%%%%%%%%%%%%%%%%%%%%%%%%%%%%%%%%%%%%%%
% Definition of black backgrounded Python code snippet
%%%%%%%%%%%%%%%%%%%%%%%%%%%%%%%%%%%%%%%%%%%%%%%%%%%%%%%%%%%%%%%%%%%%%%%%%%%
\definecolor{pink}{HTML}{ff33cc}
\definecolor{codewhite}{HTML}{ffffff}
\definecolor{codegreen}{HTML}{00e600}
\definecolor{codelemon}{HTML}{99ff99}
\definecolor{codegray}{HTML}{bfbfbf}
\definecolor{codepurple}{HTML}{9933ff}
\definecolor{codeblue}{HTML}{0099ff}
\definecolor{codered}{HTML}{ff3333}
\definecolor{bg}{HTML}{000000}

\lstdefinestyle{nightly}{
    language=Python,
    backgroundcolor=\color{bg},   
    commentstyle=\fontfamily{cmss}\color{codegreen},
    keywordstyle=\fontfamily{cmss}\color{codeblue},
    otherkeywordstyle=\fontfamily{cmss}\color{codered},            
    numberstyle=\fontfamily{cmss}\small\color{codegray},
    stringstyle=\fontfamily{cmss}\color{codepurple},
    basicstyle=\fontfamily{cmss}\small\color{codewhite},
    emph={def,is,not,in,False,True,as,and,or,from},
    emphstyle={\fontfamily{cmss}\small\color{pink}},
    emph={[2]MyClass,__init__,__repr__,__print__},
    emphstyle={[2]\fontfamily{cmss}\small\color{codered}},
    emph={[3]str,float,tuple,int,list},
    emphstyle={[3]\fontfamily{cmss}\small\color{codelemon}},
    breakatwhitespace=false,         
    breaklines=true,                 
    captionpos=b,                    
    keepspaces=true,                 
    numbers=left,                    
    numbersep=5pt,                  
    showspaces=false,                
    showstringspaces=false,
    showtabs=false,                  
    tabsize=2
}
%%%%%%%%%%%%%%%%%%%%%%%%%%%%%%%  End of definition  %%%%%%%%%%%%%%%%%%%%


%%%%%%%%%%%%%%%%%%%%%%%%%%%%%%%%%%%%%%%%%%%%%%%%%%%%%
% Titre, date, auteur
%%%%%%%%%%%%%%%%%%%%%%%%%%%%%%%%%%%%%%%%%%%%%%%%%%%%%

\author{} %\author{42-AI}
\title{Practise sheet}
%\maketitle

%%%%%%%%%%%%%%%%%%%%%%%%%%%%%%%%%%%%%%%%%%%%%%%%%%%%%
% Début du document
%%%%%%%%%%%%%%%%%%%%%%%%%%%%%%%%%%%%%%%%%%%%%%%%%%%%%
\begin{document}


%%% >>>>> Page de garde
\vspace*{2cm}
\begin{center}
    \textsc{\fontsize{40}{48} \bfseries Practise Sheet 3}\\[0.6cm]
    \textsc{\fontsize{40}{48} \bfseries Analysis}\\[0.3cm]
\end{center}
\vspace{3cm}

\begin{figure}[!h]
\center
\includegraphics[scale=0.5]{logo-42-ai.png}
\label{fig:1st_page_logo_42ai}
\end{figure}

\vspace*{2cm}
\begin{center}
    \textsc{\fontsize{32}{48} \bfseries Somme et produit généralisés, convention d'Einstein}\\[0.6cm]
\end{center}
\vspace{3cm}

\pagenumbering{gobble}
\newpage


%%% >>>>> Document body
%\begin{center}
%\chapter{\fontsize{30}{36} \selectfont Practise Sheet 1 - Les bases des fonctions}
%\end{center}
%\vspace{10mm}

\section*{Objectifs:}
L'objectif principal de cette série d'exercices est de vous entraîner à manipuler les sommes et produits généralisées ainsi que la notation d'Einstein largement utilisé en mathématiques et que l'on retrouve dans le formalisme du machine learning..






\section*{: Exercice 1:}
Dans cette exercice vous allez devoir expliciter les sommes et produits généralisées ci dessous.
\begin{align*}
    (1)\hspace{1cm} & \sum_{i = 1}^{4}2i  & (2)\hspace{1cm} & \prod_{j=1}^{5} \left(\frac{1}{2}\right)^j \\
    (3)\hspace{1cm} & \sum_{k = 1}^{5} (5-k)^k & (4)\hspace{1cm} & \prod_{l=0}^{5}\left(\frac{1}{2+l}\right)^{2l} \\
    (5)\hspace{1cm} & L(X,Y,\mathbf{\theta}) = \frac{1}{5}\sum_{i=0}^{4}\left(f(x_i,\mathbf{\theta})-y_i\right)^2  & (6)\hspace{1cm} & \prod_{n=0}^{3}(n-x)^2  \\
    (7)\hspace{1cm} & f(x) = \sum_{n = 1}^{5} \frac{f^{(n)}(x_0)}{n!}(x-x_0)^n & (8)\hspace{1cm} & L(X,Y,\mathbf{\theta}) = -\sum_{l=0}^{5} y_i\log(f(x_i,\mathbf{\theta}))
\end{align*}

Les formules $(5)$, $(7)$ et $(8)$ sont des formules que vous pourriez rencontrer un jour en faisant du ML.\\
En effet, la formule $(5)$ correspond à la formule de l'erreur quadratique moyenne (MSE), la formule $(7)$ correspond au développement en série de Taylor, utilisé lorsque l'on utilise un polynôme afin d'approximer une fonction au voisinage d'un point. Enfin, $(8)$ est une formule de fonction de coût pour un problème de classification (\textit{categorical cross-entropy}).

\section*{: Exercice 2:}
\begin{enumerate}
    \item Pour les expressions ci-dessous, développer explicitement les sommes:
    \begin{equation*}
    \sum_{i = 1}^{3}\sum_{j = 1}^{2} (i+j) \quad \quad \sum_{k = 1}^{3}\sum_{l = 0}^{2} (-1)^{k}\frac{k+l}{2^l} 
    \end{equation*}
    \item Écrire sous la forme de somme généralisée les expressions suivantes:
    \begin{fleqn}
    \begin{align*}
        \hspace{2.5cm}\mathcal{S}_1 &= 1 + 2 + 3 + 4 + 5 + 6\\
        \hspace{2.5cm}\mathcal{S}_2 &= 2 + 4 + 6 + 8 + 10 + 12 + ... + 40 + 42 \\
        \hspace{2.5cm}\mathcal{S}_3 &= a_1 + a_2 + a_3 + a_4 + a_5 \\
        \hspace{2.5cm}\mathcal{S}_3 &= 1 + 2 + 4 + 8 + 16 + 32 + 64 + 128  \\
        \hspace{2.5cm}\mathcal{S}_5 &= a_1.b_1 + a_2.b_2 + a_3.b_3 + a_4.b_4 + a_5.b_5 \\
        \hspace{2.5cm}\mathcal{S}_6 &= a_1.b_1 + a_2.b_2 + ... + a_{n-1}.b_{n-1} + a_n.b_n +2n\\
        \hspace{2.5cm}\mathcal{S}_7 &= 1 - \frac{2}{3} + \frac{4}{5} - \frac{8}{9} + \frac{16}{17} - \frac{32}{33}
    \end{align*}
    \end{fleqn}
\end{enumerate}

\section{Exercice 3:} 
Dans cet exercice, vous allez manipuler la convention de notation d'Einstein.
\begin{enumerate}
    \item Les expressions suivantes utilisent la notation d'Einstein. Explicité ces expressions dans un premier temps puis reformulez les en utilisant la somme généralisée.
    \begin{align*}
    \begin{matrix}
        (1) & b_{ii} & (i \in \{1, 2, 3, 4, 5\}),\\
        (2) & a_{1j}b_{j1} & (j \in \{1, 2, 3, 4\}),
    \end{matrix}
    \end{align*}
\end{enumerate}

\noindent\rule{\textwidth}{1pt}
Si vous continuez dans le domaine de l'IA ou que vous continuez à vous formez en mathématiques, il arrivera un jour ou vous tomberez dans un premier temps sur le symbole de Kronecker, puis celui de Levi-Civita (dans cet ordre exactement !).
\subsubsection*{Symbole de Kronecker:}
Le symbole de Kronecker est une fonction à 2 variables notée $\mathbf{\delta}$, avec les 2 variables disposées en indices (le plus souvent): $\mathbf{\delta_{ij}}$.\
La particularité de cette fonction est qu'elle vaut $1$ si les 2 variables sont égales, $0$ sinon:
\begin{equation*}
\delta_{ij} = \bigg\{ \begin{matrix} 1,  \text{ si } i = j\\ 0,  \text{ si } i \ne j\end{matrix}
\end{equation*}

\noindent\rule{\textwidth}{1pt}

\begin{align*}
\begin{matrix}
(3) & b_{ij}\delta_{ij} + a_{ij}\delta_{i,j} & (i,j \in \{1, 2, 3, 4\}),\\
(4) & a_{ij}b_{jk}\delta_{ik} & (i,j,k \in \{1, 2, 3\}),
\end{matrix}
\end{align*}

\noindent\rule{\textwidth}{1pt}
\subsubsection*{Symbole de Levi-Civita:}
Le symbole de Levi-Civita $\epsilon$ est un déterminant de matrice $(2,2)$ ou $(3,3)$ (dans la majorité des cas), usuellement noté $\varepsilon_{ij}$ ou $\varepsilon_{ijk}$ en fonction de la taille:
\begin{equation*}
\varepsilon_{ij} = 
\begin{vmatrix}
\delta_{i1} & \delta_{i2} \\
\delta_{j1} & \delta_{j2}
\end{vmatrix}\\
\varepsilon_{ijk} = 
\begin{vmatrix}
\delta_{i1} & \delta_{i2} & \delta_{i3} \\
\delta_{j1} & \delta_{j2} & \delta_{j3} \\
\delta_{k1} & \delta_{k2} & \delta_{k3}
\end{vmatrix}
\end{equation*}

Ce qui donne:

\begin{equation*}
\varepsilon_{ij} = \Bigg\{
\begin{matrix}
1, & \text{ si } & (i,j) = (1,2)\\
-1,& \text{ si } & (i,j) = (2,1)\\
0, & \text{ si } & i=j
\end{matrix}\\
\varepsilon_{ijk} = \Bigg\{ 
\begin{matrix}
1, & si & (i,j,k) \text{ est } & (1,2,3), (2,3,1) \text{ ou } (3,1,2)\\
-1, & si & (i,j,k) \text{ est } & (1,3,2), (3,2,1) \text{ ou } (2,1,3)\\
0, & si & i=j \text{, ou } & j = k \text{, ou } i = k
\end{matrix}
\end{equation*}
\textit{(Vous voyez c'était bien le symbole de Kronecker d'abord puis celui de Levi-Civita ensuite !)}

\noindent\rule{\textwidth}{1pt}


\begin{align*}
\begin{matrix}
(5) & b_{ij}\varepsilon_{ij} & (i,j \in \{1, 2, 3, 4\}), \\
(6) & b_{ij}\delta_{ij} + b_{kl}\varepsilon_{kl} & (i,j,k,l \in \{1, 2, 3\}),
\end{matrix}
\end{align*}


\end{document}
%%%%%%%%%%%%%%%%%%%%%%%%%%%%%%%%%%%%%%%%%%%%%%%%%%%%%